\PassOptionsToPackage{table,usenames,dvipsnames,x11names}{xcolor}
\documentclass[english,serif,mathserif]{beamer}
\usetheme[informal]{gc3}

\usepackage[T1]{fontenc}
\usepackage[utf8]{inputenc}
\usepackage{babel}

\usepackage{gc3}

\usepackage{colortbl}
\makeatletter
\rowcolors{1}{uzh@blue!10}{white}
\makeatother

\usepackage{dcolumn}
\newcolumntype{d}[1]{D{.}{\cdot}{#1} }


\begin{document}

\title[Spark]{An Insufficient Introduction to Spark}
\subtitle{Part 4: Relational Algebra and Table Processing}
\author{Riccardo Murri \texttt{<riccardo.murri@gmail.com>}}
\date{2019-02-11}

%% Makes the title slide
\maketitle

\part{Relational Algebra}

\begin{frame}
  \frametitle{\emph{Relational} DBs}

  All major general purpose DBMS's are based on the so-called relational
  data model.

  \+ This means that all data is stored in a number of tables (with named
  columns), such as:

  \begin{center}
    \small
    \begin{tabular}{lll}
      \textbf{usr}
             & \textbf{size}
                           & \textbf{path}                  \\
      \hline
      usr264 &          17 & /scratch/iftp/usr264/cp1.log   \\
      usr116 & 19362662400 & /scratch/id/usr116/vkeller.tar \\
      usr116 &     3379200 & /scratch/id/usr116/test.tar    \\
      usr264 &          16 & /scratch/iftp/usr264/cp2.log   \\
      usr345 &      877366 & /scratch/aim/usr345/bwa/bwa    \\
    \end{tabular}
  \end{center}

  For historical and mathematical reasons such tables are referred to as
  \emph{relations.}
\end{frame}


\begin{frame}[fragile]
  \frametitle{Relational data  model}

  \normalsize
  A \textbf{relational database} is a set of relations.

  \+
  A \textbf{relation} is an (ordered) set of tuples.

  \+
  A relation can be represented by listing all groups of related elements
  --- the result is a ``table''.

  \+
  \small
  \begin{center}
    \begin{tabular}{lll}
      \textbf{usr}
             & \textbf{size}
                           & \textbf{path}                  \\
      \hline
      usr264 &          17 & /scratch/iftp/usr264/cp1.log   \\
      usr116 & 19362662400 & /scratch/id/usr116/vkeller.tar \\
      usr116 &     3379200 & /scratch/id/usr116/test.tar    \\
      usr264 &          16 & /scratch/iftp/usr264/cp2.log   \\
      usr345 &      877366 & /scratch/aim/usr345/bwa/bwa    \\
    \end{tabular}
  \end{center}
\end{frame}


\begin{frame}
  \frametitle{What is \emph{relational} algebra?}

  Relational algebra, defined in its basic form by E.~F.~Codd in 1970, has:
  \begin{itemize}
  \item relations as atomic operands, and
  \item various operations on relations as operators (which will be detailed shortly).
  \end{itemize}

  \+
  Relational algebra is the basis of SQL and of Spark DataFrames.
\end{frame}


\begin{frame}
  \frametitle{Relational Algebra}
  \texttt{DataFrame} are basically tables: all the relational algebra operators
  that we already know can be applied.

  \+
  \begin{description}
  \item[df.select(\ldots)] return new \texttt{DataFrame} with only the
    specified set of columns (RA's ``projection'' operator).
  \item[df.distinct()] return new \texttt{DataFrame} omitting duplicate rows.
  \item[df.where(\ldots)] return new \texttt{DataFrame} containing only rows
    that satisfy a condition (RA's ``selection'' operator).
  \item[df.orderBy(\ldots)] return new \texttt{DataFrame} sorted by the specified column(s).
  \end{description}
\end{frame}


\begin{frame}
  \frametitle{Relational Algebra (2)}
  \texttt{DataFrame} are basically tables: all the relational algebra operators
  that we already know can be applied.

  \+
  \begin{description}
  \item[df1.insersect(df2)] return new \texttt{DataFrame} containing only rows
    that are in \emph{both} \texttt{df1} and \texttt{df2}
  \item[df.subtract(df2)] Return a new \texttt{DataFrame} containing only rows
    that are in \texttt{df1} \emph{but not} \texttt{df2}
  \item[df.unionAll(df2)] Return a new \texttt{DataFrame} containing union of rows.
  \end{description}
\end{frame}


\begin{frame}[fragile]
  \frametitle{Relational Algebra (3)}
  \smaller

  \texttt{DataFrame} are basically tables: all the relational algebra operators
  that we already know can be applied.

  \+
  \begin{describe}{df1.join(df2, on=$\ldots$, how=$\ldots$)}
    Perform a join of two \texttt{DataFrame}s, return result as a new \texttt{DataFrame}.

    \+
    The \texttt{how=} parameter is a string naming the type of JOIN operation:
    \texttt{'inner'}, \texttt{'outer'}, \texttt{'left\_outer'},
    \texttt{'right\_outer'}.

    \+
    The \texttt{on=} parameter is any of the following:
    \begin{itemize}
    \item \texttt{None} (default): perform a natural join
    \item column name or list of column names: perform an equi-join on the given
      columns
    \item column expression (or list thereof): perform a $\theta$-join
    \end{itemize}
  \end{describe}
\end{frame}


\part{Joins}
\begin{frame}
  \frametitle{Joins}

  From \href{https://en.wikipedia.org/wiki/Join_(SQL)}{Wikipedia}:

  \+
  \begin{quote}
    The \texttt{JOIN} operation combines columns from one or more tables in a
    relational database {\em [$\ldots$]} by using values common to each.

    \+ There are 4 common types of JOIN: \texttt{INNER},
    \texttt{LEFT~OUTER}, \texttt{RIGHT~OUTER}, and \texttt{FULL~OUTER}.
  \end{quote}

   \+
   \begin{references}
     For more details, see:
     \url{https://en.wikipedia.org/wiki/Join_(SQL)}
   \end{references}
\end{frame}


\begin{frame}[fragile]
  \frametitle{INNER JOIN}

  An \texttt{INNER JOIN} returns the set of tuples from the cross product of two
  tables that satisfy a certain predicate:

  \begin{center}
    \begin{tabular}{cc}
      \rowcolor{white}\multicolumn{2}{c}{\em Table: Directors}
      \\
      \hline
      \textbf{name} & \textbf{prizeId}
      \\
      \hline
      F.~F.~Coppola & 11
      \\
      T.~Kitano & 27
    \end{tabular}
    {\color{gray}$\pmb\Join$}
    \begin{tabular}{cc}
      \multicolumn{2}{c}{\em Table: Prizes}
      \\
      \hline
      \textbf{id} & \textbf{prize}
      \\
      \hline
      27 & Leone d'Oro
      \\
      11 & Oscar
    \end{tabular}
    \\
    {\color{gray}$\pmb\downarrow$}
    \\
    \begin{tabular}{cc}
      \multicolumn{2}{c}{\em Table: Result}
      \\
      \hline
      \textbf{name} & \textbf{prize}
      \\
      \hline
      F.~F.~Coppola & Oscar
      \\
      T.~Kitano & Leone d'Oro
      \\
    \end{tabular}

    \+
    \begin{minipage}{32ex}
      \begin{python}
directors.join(prizes,
  on=(directors.prizeId == prizes.id),
  how='inner')
    \end{python}
  \end{minipage}
  \end{center}
\end{frame}


\begin{frame}
  \frametitle{Equi-joins}

  A join operation is called an \emph{equi-join} if the selection predicate is a
  conjunction of equality comparisons.

  \+ In the case of an equi-join, the \texttt{on=\ldots} argument can be omitted.
\end{frame}


\def\ojoin{\setbox0=\hbox{$\Join$}%
  \rule[0.1ex]{.27em}{.4pt}\llap{\rule[1.3ex]{.27em}{.4pt}}}
\def\leftouterjoin{\mathbin{\ojoin\mkern-5.8mu\Join}}
\def\rightouterjoin{\mathbin{\Join\mkern-5.8mu\ojoin}}
\def\fullouterjoin{\mathbin{\ojoin\mkern-5.8mu\Join\mkern-5.8mu\ojoin}}

\begin{frame}
  \frametitle{OUTER JOIN}

  In a \texttt{OUTER JOIN}, the result table retains each row, even if no other
  matching row exists.

  \+
  \begin{itemize}
  \item In a \texttt{LEFT OUTER JOIN}, every row from the \textbf{left} table is
    retained: the result is padded with \texttt{NULL} if no matching row from
    the \emph{right} table is found.
  \item A \texttt{RIGHT OUTER JOIN} does the same with \emph{left} and
    \textbf{right} reversed.
  \item In a \texttt{FULL OUTER JOIN}, rows from both sides are retained.
  \end{itemize}
\end{frame}


\begin{frame}[fragile]
  \frametitle{LEFT OUTER JOIN}

  \begin{center}
    \begin{tabular}{cc}
      \rowcolor{white}\multicolumn{2}{c}{\em Table: Directors}
      \\
      \hline
      \textbf{name} & \textbf{prizeId}
      \\
      \hline
      F.~F.~Coppola & 11
      \\
      T.~Kitano & 27
      \\
      Ed Wood & \texttt{NULL}
    \end{tabular}
    {\color{gray}$\pmb\leftouterjoin$}
    \begin{tabular}{cc}
      \multicolumn{2}{c}{\em Table: Prizes}
      \\
      \hline
      \textbf{id} & \textbf{prize}
      \\
      \hline
      27 & Leone d'Oro
      \\
      11 & Oscar
    \end{tabular}
    \\
    {\color{gray}$\pmb\downarrow$}
    \\
    \begin{tabular}{cc}
      \multicolumn{2}{c}{\em Table: Result}
      \\
      \hline
      \textbf{name} & \textbf{prize}
      \\
      \hline
      F.~F.~Coppola & Oscar
      \\
      T.~Kitano & Leone d'Oro
      \\
      Ed Wood & \texttt{NULL}
    \end{tabular}

    \+
    \begin{minipage}{32ex}
      \begin{python}
directors.join(prizes,
  on=(directors.prizeId == prizes.id),
  how='left_outer')
    \end{python}
  \end{minipage}
  \end{center}
\end{frame}


\begin{frame}
  \begin{exercise*}[4.A]
    Re-use and adapt code from Exercise 3.B to construct DataFrames of
    word counts of files \texttt{hdfs:///shakespeare.txt.gz} and
    \texttt{hdfs:///dickens.txt.gz}.

    Then compute the top-5 words that are in Shakespeare's works and do
    not appear in Dickens, and vice-versa.
  \end{exercise*}
\end{frame}


\part{SQL queries}
\begin{frame}[fragile]
  \frametitle{SQL queries}
  It is possible to run SQL queries on \texttt{DataFrame} objects.

  \+
  \begin{describe}{df.createTempView(\emph{name})}
    Allow queries on {\ttfamily\em df} as table {\ttfamily\em name}.  Lifetime of the
    table is tied to \texttt{SQLContext}.
  \end{describe}

  \+
  \begin{describe}{spark.sql(\emph{query})}
    Run {\ttfamily\em query} and return result as a new \texttt{DataFrame}.
    All registered DFs can be queried.
    \begin{python}
df.createTempView('data')
df2 = spark.sql('SELECT * FROM data;')
    \end{python}
  \end{describe}
\end{frame}


\end{document}
%%% Local Variables:
%%% mode: latex
%%% TeX-master: t
%%% End:
